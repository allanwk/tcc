\subsection{Otimizador inicial}

A primeira parte do projeto desenvolvida foi a versão inicial do otimizador, cuja tarefa era gerar uma grade horária válida, evitando conflitos, ou seja, professores alocados para mais uma turma ao mesmo tempo. Para realizar esta tarefa, aplicaram-se os conceitos de simulated anealling, originando o otimizador representado pelo algoritmo \ref{alg:otimizadorInicial}.

\begin{algorithm}
	\caption{Otimizador de grades inicial}
	\label{alg:otimizadorInicial}
	\KwIn{Lista de professores $LP$, lista de turmas $LT$, matriz de aulas por professor por turma $MA$, temperatura inicial $TI$, Taxa de resfriamento $TR$}
	\KwOut{Grade horária de professores otimizada}
	$temperatura \leftarrow TI$\\
	$grade \leftarrow$ CriaGradeInicial$(LP, LT, MA)$\\
	$minConflitos \leftarrow$ NumeroConflitos$(grade)$\\
	\While {condição de parada não atingida} {
		\For {$passo = 0$ até $numeroPassos$} {
			$turma \leftarrow EscolheTurmaAleatoria()$\\
			$linhas \leftarrow EscolheHorariosAleatoriosValidos(sala)$\\
			$delta \leftarrow CalculaDelta(sala, linhas)$\\
			$probabilidade \leftarrow e^{-delta/temperatura}$\\
			$valorAceite \leftarrow Aleatorio(0, 1)$\\
			\If {$delta < 0$ ou $probabilidade \ge valorAceite$} {
				$PermutaProfessores(sala, linha1, linha2)$\\
				\If {NumeroConflitos$(grade)$ < $minConflitos$ } {
					Imprime$(grade)$\\
					$minConflitos \leftarrow$ NumeroConflitos$(grade)$
				}
			}
		}
		$temperatura \leftarrow temperatura * TR$
	}
\end{algorithm}

No algoritmo \ref{alg:otimizadorInicial}, o valor da taxa de resfriamento é de 0,99, sendo a temperatura inicial e o número de passos por iteração escolhidos empiricamente. Além disso, durante o desenvolvimento deste primeiro incremento, utilizou-se como condição de parada o esgotamento da temperatura, ou seja, o algoritmo finaliza sua execução assim que a temperatura atinge um valor próximo de zero, quando não ocorrem mais permutações de professores.

Explicando melhor o algoritmo, o método "CriaGradeInicial" gera uma matriz com os as turmas e números de aulas de cada professor alocados corretamente. Esta grade inicial provavelmente possui inúmeros conflitos, portanto são aplicados os passos de otimização. 

Para cada passo de otimização, são escolhidas aleatoriamente uma turma e duas linhas (posições) da grade horária. Com estas informações, é calculada a variação do número de conflitos que a permutação dos professores nas linhas escolhidas ocasionaria. De acordo com este valor de variação, é determinado se a troca dos professores deve ou não ser realizada: uma troca que diminua o número de conflitos sempre é aceita, enquanto uma troca que aumenta o número de conflitos pode ser aceita probabilistacamente, de acordo com o valor da temperatura na iteração atual.


