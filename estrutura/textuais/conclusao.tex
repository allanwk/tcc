% CONCLUSÃO--------------------------------------------------------------------

\chapter{CONSIDERAÇÕES FINAIS}
\label{chap:conclusao}

Em conclusão, este trabalho abordou o \textit{High School Timetabling Problem}, e a demanda associada por um software de otimização de grades horárias. Verificou-se durante a revisão de literatura a dificuldade que a tarefa de planejamento de grades horárias representa, e que grande parte das instituições de ensino do país ainda realiza essa tarefa manualmente.

O presente trabalho trouxe também algumas demonstrações do estado atual do protótipo da aplicação desenvolvida, assim como melhorias sugeridas para a continuação da pesquisa. Com o desenvovolvimento dos primeiros incrementos da aplicação, foi possível verificar a viabilidade do projeto aplicando a técnica de \textit{Simulated Annealing} para produzir e otimizar grades horárias escolares, atendendo múltiplos objetivos tradicionais do problema, assim como a nova preocupação trazida pelo Novo Ensino Médio, os Itinerários Formativos.

Resta portanto como atividade de crucial importância a coleta de dados de ponteciais usuários futuros da aplicação, para que seja possível verificar a necessidade de adições e melhorias ao otimizador, e a execução dos incrementos do software a fim de cumprir os objetivos propostos.