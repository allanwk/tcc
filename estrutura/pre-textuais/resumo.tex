% RESUMO--------------------------------------------------------------------------------

\begin{resumo}[RESUMO]
\begin{SingleSpacing}

% Não altere esta seção do texto--------------------------------------------------------
\imprimirautorcitacao. \imprimirtitulo. \imprimirdata. \pageref {LastPage} f. \imprimirprojeto\ – \imprimirprograma, \imprimirinstituicao. \imprimirlocal, \imprimirdata.\\
%---------------------------------------------------------------------------------------

O \textit{High School Timetabling Problem} consiste na tarefa de alocar aulas a horários letivos em instituições de ensino, considerando restrições relacionadas aos professores, salas e disciplinas. Apesar de todos os avanços tecnológicos que experenciamos na atualidade, esta tarefa ainda é realizada primordialmente de maneira manual nas escolas brasileiras, produzindo grades horárias subótimas e consequentemente trazendo frustrações a todos os envolvidos, sejam estes alunos, docentes ou diretores. Todas as dificuldades relacionadas ao problema, somadas a novos desafios como os Itinerários Formativos do Novo Ensino Médio, pedem uma solução melhor. Este trabalho teve como objetivo implementar um sistema de geração automatizada das grades horárias atendendo o maior número de objetivos possível, através da aplicação de um algoritmo de otimização baseado na meta-heurística \textit{Simulated Annealing}, técnica de otimização inspirada no processo de recozimento de materiais na metalurgia. A solução desenvolvida é uma aplicação \textit{web} que segue uma arquitetura cliente-servidor empregando JavaScript no \textit{frontend} e \textit{backend}, além do otimizador desenvolvido na linguagem C++. Espera-se que a aplicação desenvolvida passe a fazer parte do ferramental de instituições de ensino e auxilie estas a entregar horários escolares que superem os diversos desafios associados e providenciem uma experiência melhor para todos.
\\

\textbf{Palavras-chave}: High School Timetabling Problem. Simulated Annealing. Otimização multiobjetivo.

\end{SingleSpacing}
\end{resumo}

% OBSERVAÇÕES---------------------------------------------------------------------------
% Altere o texto inserindo o Resumo do seu trabalho.
% Escolha de 3 a 5 palavras ou termos que descrevam bem o seu trabalho 
