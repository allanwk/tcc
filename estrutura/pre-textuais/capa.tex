% CAPA---------------------------------------------------------------------------------------------------

% ORIENTAÇÕES GERAIS-------------------------------------------------------------------------------------
% Caso algum dos campos não se aplique ao seu trabalho, como por exemplo,
% se não houve coorientador, apenas deixe vazio.
% Exemplos: 
% \coorientador{}
% \departamento{}

% DADOS DO TRABALHO--------------------------------------------------------------------------------------
\titulo{Geração de Grades Horárias Escolares: Automação com base na meta-heurística Simulated Annealing}
\titleabstract{School Timetable Generation: Simulated Annealing based Generation}
\autor{Allan Wendland Kretzmann}
\autorcitacao{KRETZMANN, Allan} % Sobrenome em maiúsculo
\local{Cornélio Procópio}
\data{2023}

% NATUREZA DO TRABALHO-----------------------------------------------------------------------------------
% Opções: 
% - Trabalho de Conclusão de Curso (se for Graduação)
% - Dissertação (se for Mestrado)
% - Tese (se for Doutorado)
% - Projeto de Qualificação (se for Mestrado ou Doutorado)
\projeto{Trabalho de Conclusão de Curso}

% TÍTULO ACADÊMICO---------------------------------------------------------------------------------------
% Opções:
% - Bacharel ou Tecnólogo (Se a natureza for Trabalho de Conclusão de Curso)
% - Mestre (Se a natureza for Dissertação)
% - Doutor (Se a natureza for Tese)
% - Mestre ou Doutor (Se a natureza for Projeto de Qualificação)
\tituloAcademico{Bacharel em Engenharia de Computação}

% ÁREA DE CONCENTRAÇÃO E LINHA DE PESQUISA---------------------------------------------------------------
% Se a natureza for Trabalho de Conclusão de Curso, deixe ambos os campos vazios
% Se for programa de Pós-graduação, indique a área de concentração e a linha de pesquisa
\areaconcentracao{}
\linhapesquisa{}

% DADOS DA INSTITUIÇÃO-----------------------------------------------------------------------------------
% Se a natureza for Trabalho de Conclusão de Curso, coloque o nome do curso de graduação em "programa"
% Formato para o logo da Instituição: \logoinstituicao{<escala>}{<caminho/nome do arquivo>}
\instituicao{Universidade Tecnológica Federal do Paraná}
\departamento{Departamento Acadêmico de Computação}
\programa{Curso de Engenharia de Computação}
\logoinstituicao{0.2}{dados/figuras/logo-instituicao.png} 

% DADOS DOS ORIENTADORES---------------------------------------------------------------------------------
\orientador{André Yoshiaki Kashiwabara}
%\orientador[Orientadora:]{Nome da orientadora}
\instOrientador{Universidade Tecnológica Federal do Paraná}

%\coorientador{Nome do coorientador}
%\coorientador[Coorientadora:]{Nome da coorientadora}
%\instCoorientador{Instituição do coorientador}
