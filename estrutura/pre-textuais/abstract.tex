% ABSTRACT--------------------------------------------------------------------------------

\begin{resumo}[ABSTRACT]
\begin{SingleSpacing}

% Não altere esta seção do texto--------------------------------------------------------
\imprimirautorcitacao. \imprimirtitleabstract. \imprimirdata. \pageref {LastPage} f. \imprimirprojeto\ – \imprimirprograma, \imprimirinstituicao. \imprimirlocal, \imprimirdata.\\
%---------------------------------------------------------------------------------------

The High School Timetabling Problem involves the task of assigning classes to class periods in educational institutions while considering constraints related to teachers, classrooms, and subjects. Despite the technological advancements we experience today, this task is still primarily done manually in brazilian schools, resulting in suboptimal timetables and frustrations for all stakeholders, including students, teachers, and administrators. The difficulties associated with the problem, combined with new challenges provided by the new high school model implemented in the country, call for a better solution. This study has aimed to implement an automated timetabling system that satisfies as many objectives as possible by applying an optimization algorithm based on the Simulated Annealing metaheuristic, a technique inspired by the annealing process in metallurgy. The developed solution is a web application following a client-server architecture, employing JavaScript for both the frontend and backend, in addition to the optimizer implemented in the C++ language. We hope that the developed application becomes part of the toolkit for educational institutions and helps them deliver school timetables that overcome the various associated challenges and provide a better experience for all parties involved.
\\

\textbf{Keywords}: High School Timetabling Problem. Simulated Annealing. Multi-objective optimization.

\end{SingleSpacing}
\end{resumo}

% OBSERVAÇÕES---------------------------------------------------------------------------
% Altere o texto inserindo o Abstract do seu trabalho.
% Escolha de 3 a 5 palavras ou termos que descrevam bem o seu trabalho 
